\section{Introduction}

\subsection{The Amharic language (\foreignlanguage{ethiop}{'amAre~nA})}

Amharic is a Semitic language and the national language of Ethiopia (\foreignlanguage{ethiop}{'iteyo.peyA}). The majority of the 25 million or so speakers of Amharic can be found in Ethiopia\cite{csa_census_report}, but there are also speakers in a number of other countries, particularly Eritrea , Canada, the USA and Sweden. Amharic is written with a version of the Ge'ez script known as \foreignlanguage{ethiop}{fidale} (Fidel)\cite{omniglot}. Amharic is the second-most commonly spoken Semitic language in the world, after Arabic. Amharic is written left-to-right using a system that grew out of the Ge'ez script, called, in Ethiopian Semitic languages, Fidäl (\foreignlanguage{ethiop}{fidale}, "writing system", "letter", or "character" or abugida (\foreignlanguage{ethiop}{'abugidA} from the first four symbols, which gave rise to the modern linguistic term abugida. It has been the working language of courts, language of trade and everyday communications, the military, and the Ethiopian Orthodox Tewahedo Church since the late 12th century and remains the official language of Ethiopia today\cite{lingua_franca}. Most of the Ethiopian Jewish communities in Ethiopia and Israel speak Amharic. In Washington DC, Amharic became one of the six non-English languages in the Language Access Act of 2004, which allows government services and education in Amharic. Furthermore, Amharic is considered a holy language by the Rastafari religion and is widely used among its followers worldwide\cite{dc_language_access_act}. It is the most widely spoken language in the Horn of Africa. The Amharic script is an abugida, and the graphemes of the Amharic writing system are called fidel. Each character represents a consonant+vowel sequence, but the basic shape of each character is determined by the consonant, which is modified for the vowel. This is because these fidel originally represented distinct sounds, but phonological changes merged them. The citation form for each series is the consonant+ä form, i.e. the first column of the fidel. The Amharic script is included in Unicode, and glyphs are included in fonts available with major operating systems\cite{amharic_major_languages}.

\subsection{Frequency analysis}
Among so many cryptanalytic techniques, frequency analysis or frequency count is the most basic one other than brute-force, threat, blackmail, torture, and bribery. The frequency analysis is, in fact, the anatomy of a language. According to a book “Trattati in cifra” published in 1470 and written by Leone Battista Alberti, who is known as “Father of Western Cryptology”, the aspect of cryptanalysis using frequency analysis can be traced back to Al-Kindi, who is “The Philosopher of the Arabs” and author of 290 books on medicine, astronomy, mathematics, linguistics, and music. In 1987, the Arabic scientist Al-Kindi’s treatise was discovered in the Sulaimaniyyah Ottoman Archive in Istanbul and entitled “A Manuscript on Deciphering Cryptographic Message”. It is believed that this manuscript is the first ever known oldest description of cryptanalysis by frequency analysis\cite{decrypting_english_enhanced}.

\subsection{Other applications}
Knowing the frequency distribution is essential for developing optimal encoding schemes for communication mechanism like Telegraph by Morse code. To increase the speed of the communication, the code was designed so that the length of each character in Morse is approximately inverse to its frequency of occurrence in English. Thus the most common letter in English, the letter "E", has the shortest code, a single dot\cite{morse_code}.

Additionally, this analysis is essential for developing a keyboard layout with an efficient hand movement. Letter frequencies had a strong effect on the design of some keyboard layouts. The most frequent letters are on the bottom row of the Blickensderfer typewriter, and the home row of the Dvorak Simplified Keyboard\cite{bbc_qwerty}\cite{qwerty_review}. 

Furthermore, aforementioned also heavily contribute to the field of optical character recognition and in linguistics.
