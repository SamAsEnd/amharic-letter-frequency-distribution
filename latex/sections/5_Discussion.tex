\section{Discussion}

The Amharic language is the second most spoken  Semitic language and the official language is Ethiopia which the frequency distribution never been analyzed to acquire its frequency distribution. My analysis has uncovered a lot of statistics which will be instrumental in a lot of fields including but not limited to cryptanalysis, linguistics, and optical character recognition. These findings are self-explanatory but I would discuss some observations I found interesting.

I noticed the top 23 frequent monograms occurs more than 50 percent which means the more than half of every Amharic text is composed of only 23 characters.

The decline of the word separator ':' ('\foreignlanguage{ethiop}{hulate na.tebe}') and the use of space like English was also noticeable. Only a few texts I found from religious sites contain texts with the word separator ':'. But still, it's the most frequent punctuation in an Amharic text followed by '\foreignlanguage{ethiop}{,}' (comma),  '\foreignlanguage{ethiop}{;}' (semicolon) and '\foreignlanguage{ethiop}{::}'. Further, I have noticed writers chose to type two consecutive word separator (':') instead of the '\foreignlanguage{ethiop}{::}' full stop (period) which has it's own single glyph in Unicode with (0x1362) code point.

Another statics I noticed was that despite the letter ('\foreignlanguage{ethiop}{ne}') is the most frequent letter,  it's doesn't frequently occur in the first letter of a given word like the letter ('\foreignlanguage{ethiop}{ya}'). On the other hand, 83\% of the letter ('\foreignlanguage{ethiop}{ya}') occurred in the leading (first) position of the word from the total occurrence.
